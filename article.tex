\documentclass[12pt]{article}

\usepackage{article}

\usepackage{graphicx,url}

\usepackage[brazilian]{babel}
\usepackage[utf8]{inputenc}
\usepackage[T1]{fontenc}
\usepackage{lipsum}

     
\sloppy

\title{Instructions for Authors of SBC Conferences\\ Papers and Abstracts}

\author{Luciana P. Nedel\inst{1}, Rafael H. Bordini\inst{2}, Flavio Rech
  Wagner\inst{1}, Jomi F. Habner\inst{3} }


\address{Instituto de Informática -- Universidade Federal do Rio Grande do Sul
  (UFRGS)\\
  Caixa Postal 15.064 -- 91.501-970 -- Porto Alegre -- RS -- Brazil
\nextinstitute
  Department of Computer Science -- University of Durham\\
  Durham, U.K.
\nextinstitute
  Departamento de Sistemas e Computação\\
  Universidade Regional de Blumenal (FURB) -- Blumenau, SC -- Brazil
  \email{\{nedel,flavio\}@inf.ufrgs.br, R.Bordini@durham.ac.uk,
  jomi@inf.furb.br}
}

\begin{document} 

\maketitle

\begin{abstract}
  abstract
\end{abstract}
     
\begin{resumo} 
  resumo
\end{resumo}


\section{Introdução}

\lipsum[1]

\section{Referencial Teórico}

Nesta seção é apresentado os conceitos da computação em nuvem e da computação em nuvem
mobile, bem como seus modelos e arquiteturas.

\subsection{Computação em Nuvem}

\subsection{Computação em Nuvem Mobile}

\section{Metodologia}

Metodologiass

%Exemplo para colocar figura

% \begin{figure}[ht]
% \centering
% \includegraphics[width=.5\textwidth]{fig1.jpg}
% \caption{A typical figure}
% \label{fig:exampleFig1}
% \end{figure}


Bibliographic references must be unambiguous and uniform.  We recommend giving
the author names references in brackets, e.g. \cite{knuth:84},
\cite{boulic:91}, and \cite{smith:99}.

\bibliographystyle{article}
\bibliography{article}

\end{document}
