\documentclass[12pt]{article}
  
  \usepackage{article}
  
  \usepackage{graphicx,url}
  
  \usepackage[brazilian]{babel}
  \usepackage[utf8]{inputenc}
  \usepackage[T1]{fontenc}
  \usepackage{lipsum}
  
  
  \sloppy
  
  \title{Instructions for Authors of SBC Conferences\\ Papers and Abstracts}
  
  \author{Luciana P. Nedel\inst{1}, Rafael H. Bordini\inst{2}, Flavio Rech
  Wagner\inst{1}, Jomi F. Habner\inst{3} }
  
  
  \address{Instituto de Informática -- Universidade Federal do Rio Grande do Sul
  (UFRGS)\\
  Caixa Postal 15.064 -- 91.501-970 -- Porto Alegre -- RS -- Brazil
  \nextinstitute
  Department of Computer Science -- University of Durham\\
  Durham, U.K.
  \nextinstitute
  Departamento de Sistemas e Computação\\
  Universidade Regional de Blumenal (FURB) -- Blumenau, SC -- Brazil
  \email{\{nedel,flavio\}@inf.ufrgs.br, R.Bordini@durham.ac.uk,
  jomi@inf.furb.br}
  }
  
\begin{document} 

\maketitle

\begin{abstract}
  abstract
\end{abstract}

\begin{resumo} 
  resumo
\end{resumo}


\section{Introdução}

\lipsum[1]

\section{Referencial Teórico}

Nesta seção é apresentado os conceitos da computação em nuvem e da computação em nuvem
mobile, bem como seus modelos e arquiteturas.

\subsection{Computação em Nuvem}

\subsection{Computação em Nuvem Mobile}

\section{Metodologia}

Para a elaboração deste estudo, foi realizada uma revisão sistemática de literatura,
ou SRL(Systematic Literature Review), de acordo com os critérios e o modelo adotado 
por \cite{kitchenham2012}. Para isso, a revisão foi divida em três fases: planejamento,
execução e análise dos resultados, onde cada fase será melhor detalhada nas subseções seguintes.
A fase de Análise dos resultados, por comtemplar a parte principal desse trabalho, será detalhada
na próxima seção.

\subsection{Planejamento}

A fase de planejamento foi dividida em etapas e estão detalhadas nos subtópicos a seguir:

\subsubsection{Questão de pesquisa}

Este estudo responde a seguinte questão de pesquisa:

\begin{itemize}
  \item QP: Quais os principais desafios da computação em nuvem em aplicativos móveis?
\end{itemize}

\subsubsection{Definição da string de busca}

Para obtenção de resultados relevantes a partir das pesquisas a serem realizadas, foi necessário a criação de uma string de busca.
Para a criação dessa string, foi utilizado o método PICO(Population, Intervention, Comparison, Output), que foi
proposto por \cite{SANTOS2007}. O método PICO é uma maneira de facilitar o entendimento da questão de pesquisa, transformando-a
em uma string de busca.

A tabela \ref{pico} mostra como o método PICO foi definido.


\begin{table}[h]
  \centering
  % distancia entre a linha e o texto
  {\renewcommand\arraystretch{1.25}
  \begin{tabular}{ l l }
    \cline{1-1}\cline{2-2}  
    \multicolumn{1}{|p{5cm}|}{Elemento PICO} &
    \multicolumn{1}{p{8cm}|}{Argumento}
    \\  
    \cline{1-1}\cline{2-2}  
    \multicolumn{1}{|p{5cm}|}{P (Population)} &
    \multicolumn{1}{p{8cm}|}{mobile cloud computing}
    \\  
    \cline{1-1}\cline{2-2}  
    \multicolumn{1}{|p{5cm}|}{I (Intervention)} &
    \multicolumn{1}{p{8cm}|}{mobile cloud computing models, mobile cloud models, mobile cloud architecture, MCC}
    \\  
    \cline{1-1}\cline{2-2}  
    \multicolumn{1}{|p{5cm}|}{C (Comparison)} &
    \multicolumn{1}{p{8cm}|}{não se aplica}
    \\  
    \cline{1-1}\cline{2-2}  
    \multicolumn{1}{|p{5cm}|}{O (Outcome)} &
    \multicolumn{1}{p{8cm}|}{mobile cloud literature review, mobile cloud challenges, challenges}
    \\  
    \hline
  \end{tabular} }
  \caption{Definição da string de busca utilizando o método PICO} 
  \label{pico} 
  
\end{table}

Com os principais argumentos que abordam a questão de pesquisa deste estudo, foi possível a confecção da string de busca,
para uma busca mais precisa sobre o tema. Abaixo segue a primeira string definida:

(“mobile cloud computing models” OR “MCC models”) AND ("Mobile cloud models") AND ("Mobile cloud architecture") AND (“Mobile
cloud computing” OR “MCC”)

Após um refinamento da string acima, foi elaborada outra string com mais argumentos, que nos possibilitou uma maior precisão
sobre o matérial encontrado
para a escrita desta SRL. Abaixo segue a segunda string definida:

(“mobile cloud computing models” OR “MCC models”) AND ("Mobile cloud models") AND ("Mobile cloud architecture") AND (“Mobile
cloud computing” OR “MCC”) AND (“mobile computing”) AND (“mobile cloud literature review”) AND (“mobile cloud challenges”) AND
(“challenges”)

\subsubsection{Bases de busca}

Com a string de busca montada, definiu-se as bases de busca que nos retornaram artigos para a escrita dessa SRL. As bases de 
dados utilizadas para as pesquisas foram as seguintes:

\begin{itemize}
  \item IEEE Xplore (http://ieeexplore.ieee.org/Xplore/home.jsp)
  \item ACM Digital Library (https://dl.acm.org/)
  \item SpringerLink (https://link.springer.com/)
  \item ScienceDirect(http://www.sciencedirect.com/)
\end{itemize}

\subsubsection{Critérios}

Para a escolha das publicações foram feitos dois tipos de critérios, critério de inclusão e critério de exclusão.
Esse critérios foram utilizados para que os artigos selecionados fossem relevantes para para compor este estudo.

Critérios de inclusão:

\begin{itemize}
  \item CI1: A publicação deve estar escrita em inglês ou português;
  \item CI2: A publicação deve ter sido publicada a partir de 2005;
  \item CI3: A publicação deve apresentar estudos relevantes ao tema proposto nesta revisão sistemática no abstract.
\end{itemize}

Após selecionar as publicações, foi utilizado o seguinte critério para a exclusão do restante das publicações:

\begin{itemize}
  \item CE1. O abstract da publicação foge do tema proposto.
\end{itemize}

\subsection{Execução}

Após a definição da string de busca, foi realizada a execução da busca nas bases citadas anteriormente. Realizada a primeira
busca, foram encontrados diversas publicações, utilizando os critérios de inclusão e exclusão para filtrar as publicações
encontradas, foi feita a extração dos principais dados para a realização dessa SRL. Realizando essa filtragem através dos
critérios de inclusão e exclusão, obtivemos a seguinte tabela \ref{publicações} com a quantidade de publicações relevantes
encontradas em cada base:


\begin{table}[h]
 \centering
% distancia entre a linha e o texto
 {\renewcommand\arraystretch{1.25}
 \begin{tabular}{ l l }
  \cline{1-1}\cline{2-2}  
    \multicolumn{1}{|p{4.500cm}|}{\textbf{Base pesquisada}} &
    \multicolumn{1}{p{4.500cm}|}{\textbf{Publicações encontradas}}
  \\  
  \cline{1-1}\cline{2-2}  
    \multicolumn{1}{|p{4.500cm}|}{\textbf{IEEE Xplore} \centering } &
    \multicolumn{1}{p{4.500cm}|}{7 \centering }
  \\  
  \cline{1-1}\cline{2-2}  
    \multicolumn{1}{|p{4.500cm}|}{\textbf{ACM Digital Library} \centering } &
    \multicolumn{1}{p{4.500cm}|}{4 \centering }
  \\  
  \cline{1-1}\cline{2-2}  
    \multicolumn{1}{|p{4.500cm}|}{\textbf{SpringerLink} \centering } &
    \multicolumn{1}{p{4.500cm}|}{2 \centering }
  \\  
  \cline{1-1}\cline{2-2}  
    \multicolumn{1}{|p{4.500cm}|}{\textbf{ScienceDirect} \centering } &
    \multicolumn{1}{p{4.500cm}|}{1 \centering }
  \\  
  \hline

 \end{tabular} }
 \caption{Publicações relevantes encontradas} 
 \label{publicações} 
\end{table}


%Exemplo para colocar figura

% \begin{figure}[ht]
  % \centering
  % \includegraphics[width=.5\textwidth]{fig1.jpg}
  % \caption{A typical figure}
  % \label{fig:exampleFig1}
% \end{figure}

\section{Análise dos Resultados}

\section{Considerações Finais}

\bibliographystyle{article}
\bibliography{article}

\end{document}
